\documentclass[twocolumn]{aastex63}

% typography
\usepackage[T1]{fontenc}

\setlength{\parindent}{1.\baselineskip}
\newcommand{\acronym}[1]{{\small{#1}}}
\newcommand{\package}[1]{\textsl{#1}}
\newcommand{\gaia}{\textsl{Gaia}}
\newcommand{\pans}{\textsl{Pan-STARRS}}

\newcommand{\apw}[1]{{\color{blue} APW: #1}}
\newcommand{\kpc}{\ensuremath{\textrm{kpc}}}
\newcommand{\kms}{\ensuremath{\textrm{km}\,\textrm{s}^{-1}}}
\newcommand{\masyr}{\ensuremath{\textrm{mas}\,\textrm{yr}^{-1}}}
\newcommand{\feh}{\ensuremath{\textrm{[Fe/H]}}}
\newcommand{\afe}{\ensuremath{\textrm{[$\alpha$/Fe]}}}
\newcommand{\changes}[1]{{\textbf{#1}}}

% aastex parameters
% \received{not yet; THIS IS A DRAFT}
%\revised{not yet}
%\accepted{not yet}
% % Adds "Submitted to " the arguement.
% \submitjournal{ApJ}
\shorttitle{a comoving spur of gd-1}
\shortauthors{bonaca et al.}

%@arxiver{}
\usepackage{amsmath}

\begin{document}\sloppy\sloppypar\raggedbottom\frenchspacing % trust me

\title{The GD-1 stellar stream and its comoving spur lead the way to\\dynamically locating hidden objects in the Milky Way}

\correspondingauthor{Ana Bonaca}
\email{ana.bonaca@cfa.harvard.edu}

\author[0000-0002-7846-9787]{Ana Bonaca}
\affil{Center for Astrophysics | Harvard \& Smithsonian, 60 Garden Street, Cambridge, MA 02138, USA}

\author[0000-0003-2866-9403]{David W. Hogg}
\affiliation{Center for Cosmology and Particle Physics, Department of Physics, New York University}
\affiliation{Center for Data Science, New York University}
\affiliation{Max-Planck-Institut f\"ur Astronomie, Heidelberg}
\affiliation{Center for Computational Astrophysics, Flatiron Institute, 162 Fifth Avenue, NY 10010, USA}

\author[0000-0002-1590-8551]{Charlie Conroy}
\affil{Center for Astrophysics | Harvard \& Smithsonian, 60 Garden Street, Cambridge, MA 02138, USA}

\author{Phill Cargile}
\affil{Center for Astrophysics | Harvard \& Smithsonian, 60 Garden Street, Cambridge, MA 02138, USA}

\author{Nelson Caldwell}
\affil{Center for Astrophysics | Harvard \& Smithsonian, 60 Garden Street, Cambridge, MA 02138, USA}

\author{Rohan Naidu}
\affil{Center for Astrophysics | Harvard \& Smithsonian, 60 Garden Street, Cambridge, MA 02138, USA}

\author[0000-0003-0872-7098]{Adrian~M.~Price-Whelan}
\affiliation{Center for Computational Astrophysics, Flatiron Institute, 162 Fifth Avenue, NY 10010, USA}

\author{Josh Speagle}
\affil{Center for Astrophysics | Harvard \& Smithsonian, 60 Garden Street, Cambridge, MA 02138, USA}

\author{Ben Johnson}
\affil{Center for Astrophysics | Harvard \& Smithsonian, 60 Garden Street, Cambridge, MA 02138, USA}

\begin{abstract}\noindent % trust me
The longest thin stellar stream in the Milky Way halo, GD-1, has an ensemble of features that may uncover past perturbation.
Using high-resolution MMT/Hectochelle spectroscopy we show that a spur of GD-1-like stars outside of the main stream are kinematically and chemically consistent with the main stream.
In the spur, as in the main stream, GD-1 has a low intrinsic radial velocity dispersion, $\sigma_{V_r}\lesssim1.5\,\kms$, is metal-poor, $\feh\approx-2.1$, and has little intrinsic spread in the \feh\ and \afe\ abundances, which point to a common globular cluster progenitor.
At a fixed location along the stream, the median radial velocity offset between the spur and the main stream is smaller than $0.5\,\kms$, comparable to the measurement uncertainty.
A flyby of a massive, compact object can pull stars out of a stellar stream and produce features like the spur observed in GD-1.
In this scenario, radial velocity of the spur relative to the stream constrains the orbit of the perturber and its present-day location.
For GD-1, the family of acceptable perturber orbits overlaps the stellar and dark-matter debris of the Sagittarius dwarf galaxy in present-day position and velocity.
This indicates that GD-1 may have been perturbed by a globular cluster or an extremely compact dark-matter subhalo formerly associated with Sagittarius.
\end{abstract}

\keywords{%
stars:~kinematics~and~dynamics
  ---
Galaxy:~halo
  ---
Galaxy:~kinematics~and~dynamics
}

\section{Introduction}
\label{sec:intro}

The preeminent cosmological model predicts that galaxies like the Milky Way contain a myriad of non-luminous clumps of dark matter \citep[e.g.,][]{diemand2008, springel2008}.
Masses of these dark-matter subhalos are $\gtrsim4$ orders of magnitude lower than the total mass of the Milky Way, so they are expected to have a negligible effect on most stars in the Galaxy \citep[e.g.,][]{hopkins2008, donghia2010}.
However, even low-mass subhalos would leave evidence of interaction with stellar streams, the tidal debris of luminous satellites.
Numerical experiments have shown that subhalo encounters can heat up streams \citep[e.g.,][]{johnston2002, ibata2002}, produce gaps in their density profiles \citep[e.g.,][]{sgv2008,yoon2011}, and cause stream folds \citep[e.g.,][]{carlberg2009}.

Until recently, observations of stellar streams in the Milky Way were insufficient to allow robust searches for signatures of dark-matter subhalos \citep[cf.][]{carlberg2012, ibata2016}.
Now, proper motions from the \gaia\ mission \citep{gdr2} have revolutionized our ability to discover \citep[e.g.,][]{malhan2018,meingast2019} and characterize stellar streams \citep[e.g.,][]{bonaca2019b,shipp2019}.
Using \gaia\ data, \citet{pwb} studied a nearby, retrograde stellar stream GD-1 \citep{grillmair2006}, produced the cleanest map of a stream in the Milky Way and confidently identified several underdensities, as well as stars outside of the main stream \citep[see also][]{malhan2019b, deboer2019}.
\citet{bonaca2019a} created dynamical models of GD-1 which, following an encounter with a massive object, form a stream gap and an adjacent spur of stars that  quantitatively match the observed features.
With no known luminous object having approached GD-1 sufficiently close, there is a possibility that GD-1 was perturbed by a dark-matter subhalo.

\begin{figure*}
\begin{center}
\includegraphics[width=0.99\textwidth]{members.pdf}
\end{center}
\caption{We defined membership to the GD-1 stream with four selection criteria: (1) proper motion box \citep[from][not shown]{pwb}, (2) isochrone box (left), (3) small radial velocity offset from the GD-1 orbit (center), (4) low metallicity (right).
Starting with the proper motion selection, panels from left to right add selections marked with pink dashed lines and decrease membership to the number in the top right of the panel.
In each panel, the members and non-members are shown in dark and light blue, respectively.
Our final sample has 47 high-probability GD-1 members.
}
\label{fig:members}
\end{figure*}

Precise kinematic data are required to test whether the spur-and-gap feature in GD-1 was indeed formed in an interaction with a massive, dark object \citep{bonaca2019a}.
Until now, radial velocities have only been available in the main GD-1 stream \citep[][]{koposov2010,huang2019}.
In Section~\ref{sec:spec} we present the high-resolution spectroscopy from MMT/Hectochelle, which we used to define a sample of highly probable GD-1 members in the main stream and in the spur (\S\ref{sec:membership}).
These data show that the spur is kinematically aligned with the GD-1 stream (\S\ref{sec:kinematics}).
The small relative velocity between the stream and the spur can be explained within the impact scenario, but only if a perturber is on a specific set of orbits (\S\ref{sec:discussion}), which improves prospects of locating dark objects within the Milky Way purely from their interactions with stellar streams.


\section{Spectroscopy}
\label{sec:spec}

We observed the GD-1 stellar stream using MMT/Hectochelle multi-object spectrograph \citep{szentgyorgyi2011}.
Focusing on the perturbed area at $\phi_1\approx-40^\circ$ \citep[$\phi_{1,2}$ are coordinates oriented along and perpendicular to GD-1, respectively;][]{koposov2010}, we targeted 4 fields in the main stream, and 4 fields in the lower-density spur (labeled in Figure~\ref{fig:vr}, top).
Using the Gaia--PanSTARRS-1 cross-matched catalog, we selected retrograde stars as science targets, first prioritizing stars on the GD-1 main sequence, and then its red giant branch \citep[see][]{pwb}.
On average, we dedicated $\gtrsim170$ fibers to science targets per field, for a total of 1409 science spectra.
Up to 40 of the remaining fibers were used to estimate the sky emission.
We used the \texttt{RV31} filter covering the Mg~b triplet and observed each field for 2.25~hours (except for field 3 which was observed for 2~hours due to scheduling constraints).
With $3\times2$ spectral and spatial binning of the CCD pixels, we achieved a signal-to-noise ratio $S/N\approx2$ at $g=20$ and an effective resolution $R\approx32,000$.
% - $S/N>3$ cut for analysis.

The 2D spectra were reduced by \package{HSRED}~v2.1\footnote{\url{https://bitbucket.org/saotdc/hsred/}}.
This pipeline flat-fields, wavelength-calibrates with respect to ThAr lamp spectra, extracts 1D spectra and subtracts the sky emission.
We then used the \package{MINESweeper} code \citep{cargile2019} to forward-model the processed 1D spectra and infer stellar parameters, including radial velocities, [Fe/H] and [$\alpha$/Fe] abundances.
Radial velocities are measured to better than $\lesssim1\,\kms$ (median $\sigma_{V_r}=0.3\,\kms$), while typical uncertainties for [Fe/H] and [$\alpha$/Fe] are $0.07$\,dex and $0.05$\,dex, respectively.
Despite the sub-$\kms$ statistical precision, sky-emission lines show variations of $\approx1\,\kms$ across the two camera chips and between different exposures.
Our overall kinematic precision is therefore systematics-dominated at $\approx1\,\kms$, comparable to that typically achieved with Hectochelle \citep[e.g.,][]{caldwell2017}, and sufficient to resolve the GD-1 internal dispersion \citep{malhan2019}.

\section{Stream membership}
\label{sec:membership}

Since GD-1 is a retrograde, metal-poor stream, membership selection based on \gaia\ proper motions \citep{gdr2} and de-reddened \pans\ photometry \citep{sfd, ps1} is very efficient \citep[e.g.,][]{pwb}.
The left panel of Figure~\ref{fig:members} shows the color-magnitude diagram (CMD) of all targeted stars: those with GD-1-like proper motions are dark blue ($-9<\mu_{\phi_1}/\masyr<-4.5$ and $-1.7<\mu_{\phi_2}/\masyr<1$, corrected for solar reflex motion following \citealt{pwb}), and the filler retrograde stars are light blue.
We further consider stars close to the $\textrm{[Fe/H]}=-1.5$, 12.6\,Gyr isochrone at 7.8\,kpc \citep{choi2016} as more likely GD-1 members.
The isochrone selection box (dashed pink) is tighter around the GD-1's main sequence where the contrast with respect to the Milky Way field is higher, and more generous at the red giant branch, allowing for uncertainties in the adopted isochrone.
A total of 82 stars satisfy the proper motion and CMD selection criteria.

Most of these likely members have small radial velocity offsets from the GD-1's orbit (Figure~\ref{fig:members}, middle panel).
We used the orbital solution from \citet{pwb}, which is $\Delta V_r\approx10\,\kms$ offset from Hectochelle radial velocities, so we further select 50 more likely members with velocity offsets $|\Delta V_r| < 7\,\kms$.
Stars from both the main stream and the spur satisfy these criteria, indicating that the spur is dynamically associated with GD-1.
We explore this connection in more detail in Section~\ref{sec:kinematics}.
% 
Thus selected GD-1 members are predominantly metal-poor, as found by \citet{malhan2019}, so we adopt $-2.8<\feh<1.9$ as our final criterion for GD-1 membership (Figure~\ref{fig:members}, right).
Although wide, this metallicity selection removes only two kinematically identified members for a final sample of 47 most likely GD-1 stars.
-- membership binary here: probabilistically way to go in the future
- whole spec sample available on zenodo w separate flags

In addition to determining membership, chemical abundances of stream stars are the best way to differentiate between a globular cluster and a dwarf galaxy origin in the absence of a progenitor.
Abundance spreads are typically a signature of a dwarf galaxy origin \citep[e.g.,][]{willman2012}, however, globular clusters observed with the same instrumental setup as used here (although at a higher signal-to-noise ratio) show spreads on the order of $\approx0.05-0.1$\,dex \citep{cargile2019}.
The standard deviations in $\feh$ and $\afe$ of GD-1 members are 0.17\,dex and 0.16\,dex, respectively, and are almost entirely accounted for by the measurement uncertainties (median $\sigma_\feh=0.16$\,dex, $\sigma_\afe=0.14$\,dex, larger than our typical target since most of the members are faint main-sequence stars).
These data suggest that GD-1 is likely a disrupted globular cluster, as previously inferred from its narrow width \citep[e.g.,][]{grillmair2006} and cold kinematics \citep[e.g.,][]{malhan2019}.
More detailed abundances will enable an even more efficient membership selection, and a definitive classification of GD-1's progenitor.
-- chemistry: spread similar to gc, check w multipop (old)


\section{GD-1 kinematics}
\label{sec:kinematics}

\begin{figure}
\begin{center}
\includegraphics[width=0.99\columnwidth]{gd1_kinematics.pdf}
\end{center}
\caption{Sky positions of spectroscopically identified GD-1 members overplotted on the map of likely stream members (top panel; gray for literature data, orange for this work, with light and dark symbols for the main stream and the spur, respectively).
Both data sets agree that GD-1 has a steep radial velocity gradient (second panel), which puts tight constraints on the stream's orbit (teal).
Our high-resolution measurements show little dispersion with respect to the best-fit orbit (third panel).
At a fixed location along the stream, the median radial velocities of the main stream and the spur are consistent at a level of $\lesssim1\,\kms$ (black-edged symbols, fourth panel).
}
\label{fig:vr}
\end{figure}

We summarize radial velocity structure of the GD-1 stream in Figure~\ref{fig:vr}.
The top panel shows the on-sky distribution of likely GD-1 members identified using \gaia\ proper motions \citep[small points,][]{pwb}, and highlights stars with a measured radial velocity \citep[orange for this work, gray for literature data from][]{koposov2010}.
The second panel shows radial velocity as a function of the $\phi_1$ stream coordinate.
Our data include the first radial velocity measurements in the GD-1 spur (dark orange), and they are consistent with radial velocities in the main GD-1 stream (light orange), further supporting the case for their common origin.
Our measurements show a strong radial velocity gradient along the stream that is largely consistent with, but somewhat offset from the literature measurements (obtained at a lower-resolution).
We next search for orbits that fit the updated sample of GD-1 radial velocities.

We adopted the GD-1 orbit-fitting procedure from \citet{pwb}, including their fixed Milky Way model similar to \citet{bovy2015} and their compilation of 6D stream data which we augmented with more precise radial velocities from this work.
The radial velocity gradient of the best-fit orbit is shown with a teal line in the second panel of Figure~\ref{fig:vr}.
The best-fit orbit has a pericenter at 13.8\,kpc and an apocenter at 21.5\,kpc, making the updated orbital solution slightly more circular, but otherwise similar to the orbit derived in \citet{pwb}.

In the third and the fourth panel of Figure~\ref{fig:vr} we show the radial velocity offsets from the best-fit GD-1 orbit (teal).
Overall, our high-resolution measurements show little deviation from the orbital velocity and reveal a kinematically cold stream with a much lower dispersion than previously measured (third panel).
Accounting for measurement uncertainties, the intrinsic velocity dispersion is x in the main stream and y in the spur.
- consistent w much colder than we're seeing
- possible not reaching pure photon limit due to hectochelle systematics
- but, more comfortable getting higher-res spectroscopy
In a multi-year study of stars in the field of the MW globular cluster NGC2419 using MMT/Hectochelle with the same setup as used here, repeated measurements of stars indicate an rms velocity measurement of $0.6\,\kms$ can be achieved, which is similar to the mean uncertainty derived for individual measurements (Caldwell, Freeman, Walker in preparation).

To quantify the relative motion between the stream and the spur, we compare the median radial velocity of GD-1 members observed in individual Hectochelle fields (large symbols with black edges in the fourth panel, the errorbars are the standard deviation).
At two locations where we observed the main stream and the spur in parallel ($\phi_1=-33.7^\circ, -30^\circ$), the relative radial velocity is smaller than $1\,\kms$ and comparable to the measurement uncertainty.
To a high degree, the GD-1 spur is comoving with the stream, which puts strong constraints on its formation scenarios.


\section{Discussion}
\label{sec:discussion}

We presented high-resolution spectroscopy at eight locations in the GD-1 stellar stream, distributed along the main stream and an adjacent spur.
With the goal of discerning the association between the stream and the spur, we obtained the most precise radial velocities of GD-1 to date (statistical uncertainty $\lesssim0.5\,\kms$).
Serendipitously, these data update the GD-1's orbit (\S\,\ref{sec:kinematics}), and will improve constraints on the Milky Way's gravitational potential in future modeling of GD-1 \citep[similar to, e.g.,][]{koposov2010, bowden2015, bovy2016}.
The relative radial velocity between the stream and the spur is small, $\Delta V_r\lesssim1\,\kms$, which, combined with their similar metallicity, $\feh\approx-2.2$, suggests that the spur is a part of GD-1 that has been perturbed from an original orbit along the stream.
We conclude with a discussion of implications that a comoving spur places on its formation mechanism and an outlook for dynamical inferences about the structure of the Milky Way if features like the GD-1 spur are common in other streams.

\citet{bonaca2019a} showed that a stream can develop the spur-and-gap morphology similar to that observed in GD-1 following an encounter with a massive object, and predicted that properties of the encounter can be further constrained with kinematic data.
We next revisit perturbed models of GD-1 from \citet{bonaca2019a} and explore implications of the observed GD-1 kinematics in the context of an encounter scenario.
A spur comoving with the stream prefers models of a closer encounter with a less massive and more compact object than inferred from the stream morphology alone, while the range of allowed impact times and the perturber's total velocity remain similar.
The most substantial improvement that the kinematic data provide is in constraining the perturber's orbit, which determines its present-day location.
In the top of Figure~\ref{fig:skybox} we show the present-day sky positions of perturber models allowed by stream morphology, color-coded by the relative stream--spur radial velocity, $\Delta V_r(\phi_1=-33.7^\circ)$.
Morphology alone allows for a perturber on a variety of orbits, however, models satisfying the observed relative velocity $|\Delta V_r|<1\,\kms$ are spatially constrained (Figure~\ref{fig:skybox}, middle).

\begin{figure}
\begin{center}
\includegraphics[width=0.99\columnwidth]{skybox.pdf}
\end{center}
\caption{Present-day sky positions of objects that induce a spur-and-gap morphology after a close encounter with the GD-1 stream, color-coded by the relative radial velocity between the stream and the spur at $\phi_1=-33.7^\circ$ (top, from \citealt{bonaca2019a}).
Solutions that satisfy the measured radial velocity offset, $|\Delta V_r|<1\kms$, are approximately on a great circle (middle), coincident with the distribution of dark matter expected from the disruption of the Sagittarius dwarf galaxy (gray points, bottom, from \citealt{dl2017}).
}
\label{fig:skybox}
\end{figure}

Even with radial velocity constraints, the position of the GD-1's perturber is not entirely localized on the sky, but is instead restricted to lie approximately along a great circle.
In general, interaction models predict a radial velocity gradient along the spur \citep{bonaca2019a}, which should tighten the orbital constraints.
Radial velocities we measured in the GD-1 spur show a tentative gradient between $\phi_1=-35^\circ$ and $-31^\circ$ (Figure~\ref{fig:vr}, bottom), however, higher precision is required to resolve the gradient.
The color-coding in the middle panel of Figure~\ref{fig:skybox} maps to the relative proper motion between the GD-1 stream and its spur, indicating that precise transverse velocities from future \gaia\ data releases or targeted astrometric followup will further limit the perturber's location.

Interestingly, currently allowed positions of the GD-1 perturber spatially overlap with the expected distribution of dark matter stripped from the Sagittarius dwarf galaxy (gray points in the bottom of Figure~\ref{fig:skybox}, from \citealt{dl2017}).
A subset of GD-1 solutions between $\rm R.A.=180^\circ$ and $300^\circ$ further coincide with the expected distances and radial velocities of the Sagittarius debris.
This overlap of the 4-dimensional position and velocity distributions suggests that the GD-1 stream might have been perturbed by an object originally associated with Sagittarius.
Regardless of the nature of the perturber, its possible association with Sagittarius suggests the need for updated forecasts of gap occurrence rate in stellar streams, which have so far assumed isotropic distribution of dark-matter subhalos \citep[e.g.,][]{erkal2016, banik2019}.

The Sagittarius dwarf itself is too massive to produce the spur-and-gap morphology of GD-1, but a compact, low-mass object associated with Sagittarius, like a globular cluster or a dark-matter subhalo, is a plausible culprit.
Globular clusters appear more likely candidates due to the compact size we infer for the perturber, but none of the known clusters come closer than 1\,kpc to the GD-1 impact site during the past 2\,Gyr (based on the analysis from \citealt{bonaca2019a} with the updated orbit of GD-1 and the 6-dimensional cluster positions from \citealt{baumgardt2019}).
Still, the scenario in which GD-1 was perturbed by a globular cluster needs to be further tested, as the census of globular clusters may be incomplete and the true gravitational potential likely deviates from the idealized model we used so far.
If a luminous perturber is conclusively ruled out after these considerations have been taken into account, a dark-matter subhalo associated with Sagittarius remains a viable perturber, and its inferred high density might signal a self-interacting, instead of cold, dark matter \citep[e.g.,][]{kahlhoefer2019}.

GD-1 in the \gaia\ era is a stream with a whole host of unusual features, which sparked a discussion of additional processes to explain different aspects of the data.
For example, \citet{deboer2019} explored models in which GD-1 is perturbed by the Sagittarius dwarf galaxy.
A strong interaction with Sagittarius can launch a long spur that remains closely aligned with GD-1 before detaching from the main stream (which reproduces the widening of GD-1 at $\phi_1\lesssim-45^\circ$).
On the other hand, \citet{malhan2019c} discovered a low surface-brightness stream, Kshir, that intersects GD-1 at $\phi_1\approx-20^\circ$.
This cross-point is sufficiently close to the spur-and-gap feature that Kshir might have affected their formation.
Alternatively, \citet{webb2019} suggest that the gap at $\phi_1\approx-40^\circ$ may not be a signature of an impact, but rather the location of the GD-1's progenitor final disruption.
In that case, the spur could be a result of substructure in the progenitor \citep[e.g.,][]{carlberg2018}, instead of forming through an external perturbation.
As these ideas are developed to more closely match the observed morphology of GD-1, the kinematic data presented here will be essential in quantifying their role in the dynamical history of GD-1.

Excitingly, GD-1 is only one of several streams that display peculiar morphology \citep[e.g.,][]{pwb, bonaca2019b, bonaca2020}.
Our results imply that if these features are a result of an impact to the stream, precise kinematics of the perturbed region will pinpoint the perpetrator.
The prospect of localization would revolutionize the study of dark matter in the Milky Way.
Instead of inferring the nature of dark matter through the total abundance \citep[e.g.,][]{carlberg2013} or the mass function of dark-matter subhalos \citep[e.g.,][]{banik2019}, multi-wavelength observations of individual subhalo candidates would enable direct tests of different dark matter models \citep[e.g.,][]{daylan2016}, and add dark matter to the domain of multi-messenger astronomy.

It is a pleasure to thank Doug Finkbeiner, Lars Hernquist, Sean Moran, and Hans-Walter Rix for valuable discussions.

Observations reported here were obtained at the MMT Observatory, a joint facility of the Smithsonian Institution and the University of Arizona.

This project was developed in part at the 2019 Santa Barbara Gaia Sprint, hosted by the Kavli Institute for Theoretical Physics at the University of California, Santa Barbara.

This research was supported in part at KITP by the Heising-Simons Foundation and the National Science Foundation under Grant No. NSF PHY-1748958.

\facility{MMT(Hectochelle)}

\software{
\package{Astropy} \citep{astropy, astropy:2018},
\package{gala} \citep{gala},
\package{IPython} \citep{ipython},
\package{matplotlib} \citep{mpl},
\package{numpy} \citep{numpy},
\package{scipy} \citep{scipy}
}

\bibliographystyle{aasjournal}
\bibliography{spur_rv}


\end{document}